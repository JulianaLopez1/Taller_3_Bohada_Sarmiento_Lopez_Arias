% Options for packages loaded elsewhere
\PassOptionsToPackage{unicode}{hyperref}
\PassOptionsToPackage{hyphens}{url}
%
\documentclass[
]{article}
\usepackage{amsmath,amssymb}
\usepackage{iftex}
\ifPDFTeX
  \usepackage[T1]{fontenc}
  \usepackage[utf8]{inputenc}
  \usepackage{textcomp} % provide euro and other symbols
\else % if luatex or xetex
  \usepackage{unicode-math} % this also loads fontspec
  \defaultfontfeatures{Scale=MatchLowercase}
  \defaultfontfeatures[\rmfamily]{Ligatures=TeX,Scale=1}
\fi
\usepackage{lmodern}
\ifPDFTeX\else
  % xetex/luatex font selection
\fi
% Use upquote if available, for straight quotes in verbatim environments
\IfFileExists{upquote.sty}{\usepackage{upquote}}{}
\IfFileExists{microtype.sty}{% use microtype if available
  \usepackage[]{microtype}
  \UseMicrotypeSet[protrusion]{basicmath} % disable protrusion for tt fonts
}{}
\makeatletter
\@ifundefined{KOMAClassName}{% if non-KOMA class
  \IfFileExists{parskip.sty}{%
    \usepackage{parskip}
  }{% else
    \setlength{\parindent}{0pt}
    \setlength{\parskip}{6pt plus 2pt minus 1pt}}
}{% if KOMA class
  \KOMAoptions{parskip=half}}
\makeatother
\usepackage{xcolor}
\usepackage[margin=1in]{geometry}
\usepackage{color}
\usepackage{fancyvrb}
\newcommand{\VerbBar}{|}
\newcommand{\VERB}{\Verb[commandchars=\\\{\}]}
\DefineVerbatimEnvironment{Highlighting}{Verbatim}{commandchars=\\\{\}}
% Add ',fontsize=\small' for more characters per line
\usepackage{framed}
\definecolor{shadecolor}{RGB}{248,248,248}
\newenvironment{Shaded}{\begin{snugshade}}{\end{snugshade}}
\newcommand{\AlertTok}[1]{\textcolor[rgb]{0.94,0.16,0.16}{#1}}
\newcommand{\AnnotationTok}[1]{\textcolor[rgb]{0.56,0.35,0.01}{\textbf{\textit{#1}}}}
\newcommand{\AttributeTok}[1]{\textcolor[rgb]{0.13,0.29,0.53}{#1}}
\newcommand{\BaseNTok}[1]{\textcolor[rgb]{0.00,0.00,0.81}{#1}}
\newcommand{\BuiltInTok}[1]{#1}
\newcommand{\CharTok}[1]{\textcolor[rgb]{0.31,0.60,0.02}{#1}}
\newcommand{\CommentTok}[1]{\textcolor[rgb]{0.56,0.35,0.01}{\textit{#1}}}
\newcommand{\CommentVarTok}[1]{\textcolor[rgb]{0.56,0.35,0.01}{\textbf{\textit{#1}}}}
\newcommand{\ConstantTok}[1]{\textcolor[rgb]{0.56,0.35,0.01}{#1}}
\newcommand{\ControlFlowTok}[1]{\textcolor[rgb]{0.13,0.29,0.53}{\textbf{#1}}}
\newcommand{\DataTypeTok}[1]{\textcolor[rgb]{0.13,0.29,0.53}{#1}}
\newcommand{\DecValTok}[1]{\textcolor[rgb]{0.00,0.00,0.81}{#1}}
\newcommand{\DocumentationTok}[1]{\textcolor[rgb]{0.56,0.35,0.01}{\textbf{\textit{#1}}}}
\newcommand{\ErrorTok}[1]{\textcolor[rgb]{0.64,0.00,0.00}{\textbf{#1}}}
\newcommand{\ExtensionTok}[1]{#1}
\newcommand{\FloatTok}[1]{\textcolor[rgb]{0.00,0.00,0.81}{#1}}
\newcommand{\FunctionTok}[1]{\textcolor[rgb]{0.13,0.29,0.53}{\textbf{#1}}}
\newcommand{\ImportTok}[1]{#1}
\newcommand{\InformationTok}[1]{\textcolor[rgb]{0.56,0.35,0.01}{\textbf{\textit{#1}}}}
\newcommand{\KeywordTok}[1]{\textcolor[rgb]{0.13,0.29,0.53}{\textbf{#1}}}
\newcommand{\NormalTok}[1]{#1}
\newcommand{\OperatorTok}[1]{\textcolor[rgb]{0.81,0.36,0.00}{\textbf{#1}}}
\newcommand{\OtherTok}[1]{\textcolor[rgb]{0.56,0.35,0.01}{#1}}
\newcommand{\PreprocessorTok}[1]{\textcolor[rgb]{0.56,0.35,0.01}{\textit{#1}}}
\newcommand{\RegionMarkerTok}[1]{#1}
\newcommand{\SpecialCharTok}[1]{\textcolor[rgb]{0.81,0.36,0.00}{\textbf{#1}}}
\newcommand{\SpecialStringTok}[1]{\textcolor[rgb]{0.31,0.60,0.02}{#1}}
\newcommand{\StringTok}[1]{\textcolor[rgb]{0.31,0.60,0.02}{#1}}
\newcommand{\VariableTok}[1]{\textcolor[rgb]{0.00,0.00,0.00}{#1}}
\newcommand{\VerbatimStringTok}[1]{\textcolor[rgb]{0.31,0.60,0.02}{#1}}
\newcommand{\WarningTok}[1]{\textcolor[rgb]{0.56,0.35,0.01}{\textbf{\textit{#1}}}}
\usepackage{graphicx}
\makeatletter
\def\maxwidth{\ifdim\Gin@nat@width>\linewidth\linewidth\else\Gin@nat@width\fi}
\def\maxheight{\ifdim\Gin@nat@height>\textheight\textheight\else\Gin@nat@height\fi}
\makeatother
% Scale images if necessary, so that they will not overflow the page
% margins by default, and it is still possible to overwrite the defaults
% using explicit options in \includegraphics[width, height, ...]{}
\setkeys{Gin}{width=\maxwidth,height=\maxheight,keepaspectratio}
% Set default figure placement to htbp
\makeatletter
\def\fps@figure{htbp}
\makeatother
\setlength{\emergencystretch}{3em} % prevent overfull lines
\providecommand{\tightlist}{%
  \setlength{\itemsep}{0pt}\setlength{\parskip}{0pt}}
\setcounter{secnumdepth}{-\maxdimen} % remove section numbering
\ifLuaTeX
  \usepackage{selnolig}  % disable illegal ligatures
\fi
\usepackage{bookmark}
\IfFileExists{xurl.sty}{\usepackage{xurl}}{} % add URL line breaks if available
\urlstyle{same}
\hypersetup{
  pdftitle={Empirical Project 1: Working in R code},
  hidelinks,
  pdfcreator={LaTeX via pandoc}}

\title{Empirical Project 1: Working in R code}
\author{}
\date{\vspace{-2.5em}}

\begin{document}
\maketitle

\section{Empirical Project 1 Working in
R}\label{empirical-project-1-working-in-r}

These code downloads have been constructed as supplements to the full
Doing Economics projects (\url{https://core-econ.org/doing-economics/}).
You'll need to download the data before running the code that follows.

\subsection{Part 1.1 The behaviour of average surface temperature over
time}\label{part-1.1-the-behaviour-of-average-surface-temperature-over-time}

\subsubsection{R walk-through 1.1 Importing the datafile into
R}\label{r-walk-through-1.1-importing-the-datafile-into-r}

We want to import the datafile called `NH.Ts+dSST.csv' into R.

We start by setting our working directory using the \texttt{setwd}
command. This command tells R where your datafiles are stored. In the
code below, replace `YOURFILEPATH' with the full filepath that indicates
the folder in which you have saved the datafile. If you don't know how
to find the path to your working folder, see the `Technical Reference'
section (\url{https://tinyco.re/3407438}).

\begin{Shaded}
\begin{Highlighting}[]
\FunctionTok{getwd}\NormalTok{()}
\end{Highlighting}
\end{Shaded}

\begin{verbatim}
## [1] "C:/Users/moral/OneDrive/Escritorio/Taller_3_Bohada_Sarmiento_Lopez_Arias/Code"
\end{verbatim}

\begin{Shaded}
\begin{Highlighting}[]
\FunctionTok{setwd}\NormalTok{(}\StringTok{"C:/Users/moral/OneDrive/Escritorio/Taller\_3\_Bohada\_Sarmiento\_Lopez\_Arias"}\NormalTok{)}
\end{Highlighting}
\end{Shaded}

Since our data is in csv format, we use the \texttt{read.csv} function
to import the data into R. We will call our file `tempdata' (short for
`temperature data').

Here you can see commands to R which are spread across two lines. You
can spread a command across multiple lines, but you must adhere to the
following two rules for this to work. First, the line break should come
inside a set of parenthesis (i.e.~between \texttt{(} and \texttt{)} or
straight after the assignment operator (\texttt{\textless{}-}). Second,
the line break must not be inside a string (whatever is inside quotes)
or in the middle of a word or number.

\begin{Shaded}
\begin{Highlighting}[]
\NormalTok{tempdata }\OtherTok{\textless{}{-}} \FunctionTok{read.csv}\NormalTok{(}\StringTok{"C:/Users/moral/OneDrive/Escritorio/Taller\_3\_Bohada\_Sarmiento\_Lopez\_Arias/RAW/NH.Ts+dSST.csv"}\NormalTok{,}
  \AttributeTok{skip =} \DecValTok{1}\NormalTok{, }\AttributeTok{na.strings =} \StringTok{"***"}\NormalTok{) }
\end{Highlighting}
\end{Shaded}

When using this function, we added two options. If you open the
spreadsheet in Excel, you will see that the real data table only starts
in Row 2, so we use the \texttt{skip\ =\ 1} option to skip the first row
when importing the data. When looking at the spreadsheet, you can see
that missing temperature data is coded as \texttt{"***"}. In order for R
to recognise the non-missing temperature data as numbers, we use the
\texttt{na.strings\ =\ "***"} option to indicate that missing
observations in the spreadsheet are coded as \texttt{"***"}.

To check that the data has been imported correctly, you can use the
\texttt{head} function to view the first six rows of the dataset, and
confirm that they correspond to the columns in the csv file.

\begin{Shaded}
\begin{Highlighting}[]
\FunctionTok{head}\NormalTok{(tempdata)}
\end{Highlighting}
\end{Shaded}

\begin{verbatim}
##   Year   Jan   Feb   Mar   Apr   May   Jun   Jul   Aug   Sep   Oct   Nov   Dec
## 1 1880 -0.38 -0.52 -0.23 -0.30 -0.04 -0.18 -0.21 -0.25 -0.24 -0.29 -0.43 -0.41
## 2 1881 -0.30 -0.24 -0.05 -0.02  0.05 -0.33  0.10 -0.05 -0.27 -0.44 -0.36 -0.23
## 3 1882  0.26  0.22  0.03 -0.29 -0.22 -0.28 -0.28 -0.14 -0.24 -0.51 -0.33 -0.67
## 4 1883 -0.57 -0.65 -0.14 -0.29 -0.25 -0.11 -0.05 -0.22 -0.33 -0.15 -0.43 -0.14
## 5 1884 -0.15 -0.10 -0.63 -0.58 -0.35 -0.40 -0.40 -0.50 -0.44 -0.43 -0.57 -0.46
## 6 1885 -1.00 -0.45 -0.23 -0.48 -0.57 -0.44 -0.33 -0.40 -0.38 -0.36 -0.37 -0.10
##     J.D   D.N   DJF   MAM   JJA   SON
## 1 -0.29    NA    NA -0.19 -0.21 -0.32
## 2 -0.18 -0.19 -0.32 -0.01 -0.09 -0.36
## 3 -0.20 -0.17  0.08 -0.16 -0.23 -0.36
## 4 -0.28 -0.32 -0.63 -0.23 -0.12 -0.31
## 5 -0.42 -0.39 -0.13 -0.52 -0.44 -0.48
## 6 -0.43 -0.46 -0.64 -0.42 -0.39 -0.37
\end{verbatim}

Before working with the important data, we use the \texttt{str} function
to check that the data is formatted correctly.

\begin{Shaded}
\begin{Highlighting}[]
\FunctionTok{str}\NormalTok{(tempdata)}
\end{Highlighting}
\end{Shaded}

\begin{verbatim}
## 'data.frame':    145 obs. of  19 variables:
##  $ Year: int  1880 1881 1882 1883 1884 1885 1886 1887 1888 1889 ...
##  $ Jan : num  -0.38 -0.3 0.26 -0.57 -0.15 -1 -0.73 -1.08 -0.48 -0.27 ...
##  $ Feb : num  -0.52 -0.24 0.22 -0.65 -0.1 -0.45 -0.83 -0.7 -0.61 0.3 ...
##  $ Mar : num  -0.23 -0.05 0.03 -0.14 -0.63 -0.23 -0.7 -0.44 -0.63 -0.01 ...
##  $ Apr : num  -0.3 -0.02 -0.29 -0.29 -0.58 -0.48 -0.36 -0.38 -0.21 0.17 ...
##  $ May : num  -0.04 0.05 -0.22 -0.25 -0.35 -0.57 -0.33 -0.25 -0.14 -0.03 ...
##  $ Jun : num  -0.18 -0.33 -0.28 -0.11 -0.4 -0.44 -0.37 -0.2 -0.02 -0.06 ...
##  $ Jul : num  -0.21 0.1 -0.28 -0.05 -0.4 -0.33 -0.14 -0.24 0 -0.08 ...
##  $ Aug : num  -0.25 -0.05 -0.14 -0.22 -0.5 -0.4 -0.42 -0.55 -0.21 -0.2 ...
##  $ Sep : num  -0.24 -0.27 -0.24 -0.33 -0.44 -0.38 -0.32 -0.2 -0.19 -0.29 ...
##  $ Oct : num  -0.29 -0.44 -0.51 -0.15 -0.43 -0.36 -0.31 -0.49 -0.03 -0.41 ...
##  $ Nov : num  -0.43 -0.36 -0.33 -0.43 -0.57 -0.37 -0.4 -0.27 0 -0.61 ...
##  $ Dec : num  -0.41 -0.23 -0.67 -0.14 -0.46 -0.1 -0.21 -0.42 -0.23 -0.54 ...
##  $ J.D : num  -0.29 -0.18 -0.2 -0.28 -0.42 -0.43 -0.43 -0.43 -0.23 -0.17 ...
##  $ D.N : num  NA -0.19 -0.17 -0.32 -0.39 -0.46 -0.42 -0.42 -0.25 -0.14 ...
##  $ DJF : num  NA -0.32 0.08 -0.63 -0.13 -0.64 -0.55 -0.66 -0.5 -0.07 ...
##  $ MAM : num  -0.19 -0.01 -0.16 -0.23 -0.52 -0.42 -0.46 -0.35 -0.33 0.04 ...
##  $ JJA : num  -0.21 -0.09 -0.23 -0.12 -0.44 -0.39 -0.31 -0.33 -0.07 -0.11 ...
##  $ SON : num  -0.32 -0.36 -0.36 -0.31 -0.48 -0.37 -0.34 -0.32 -0.07 -0.44 ...
\end{verbatim}

You can see that all variables are formatted as numerical data
(\texttt{num}), so R correctly recognises that the data are numbers.

{[}End of walk-through{]}

\subsubsection{R walk-through 1.2 Drawing a line chart of temperature
and
time}\label{r-walk-through-1.2-drawing-a-line-chart-of-temperature-and-time}

The data is formatted as numerical (\texttt{num}) data, so R recognises
each variable as a series of numbers (instead of text), but does not
recognise that these numbers correspond to the same variable for
different time periods (known as `time series data' in economics).
Letting R know that we have time series data will make coding easier
later (especially with making graphs). You can use the \texttt{ts}
function to specify that a variable is a time series. Make sure to amend
the code below so that the end year (\texttt{end\ =\ c()}) corresponds
to the latest year in your dataset (our example uses 2017).

\begin{Shaded}
\begin{Highlighting}[]
\NormalTok{tempdata}\SpecialCharTok{$}\NormalTok{Jan }\OtherTok{\textless{}{-}} \FunctionTok{ts}\NormalTok{(tempdata}\SpecialCharTok{$}\NormalTok{Jan, }
  \AttributeTok{start =} \FunctionTok{c}\NormalTok{(}\DecValTok{1880}\NormalTok{),}\AttributeTok{end =} \FunctionTok{c}\NormalTok{(}\DecValTok{2024}\NormalTok{), }\AttributeTok{frequency =} \DecValTok{1}\NormalTok{) }
\NormalTok{tempdata}\SpecialCharTok{$}\NormalTok{DJF }\OtherTok{\textless{}{-}} \FunctionTok{ts}\NormalTok{(tempdata}\SpecialCharTok{$}\NormalTok{DJF, }
  \AttributeTok{start =} \FunctionTok{c}\NormalTok{(}\DecValTok{1880}\NormalTok{), }\AttributeTok{end =} \FunctionTok{c}\NormalTok{(}\DecValTok{2024}\NormalTok{), }\AttributeTok{frequency =} \DecValTok{1}\NormalTok{)}
\NormalTok{tempdata}\SpecialCharTok{$}\NormalTok{MAM }\OtherTok{\textless{}{-}} \FunctionTok{ts}\NormalTok{(tempdata}\SpecialCharTok{$}\NormalTok{MAM, }
  \AttributeTok{start =} \FunctionTok{c}\NormalTok{(}\DecValTok{1880}\NormalTok{),}\AttributeTok{end =} \FunctionTok{c}\NormalTok{(}\DecValTok{2024}\NormalTok{), }\AttributeTok{frequency =} \DecValTok{1}\NormalTok{)}
\NormalTok{tempdata}\SpecialCharTok{$}\NormalTok{JJA }\OtherTok{\textless{}{-}} \FunctionTok{ts}\NormalTok{(tempdata}\SpecialCharTok{$}\NormalTok{JJA, }
  \AttributeTok{start =} \FunctionTok{c}\NormalTok{(}\DecValTok{1880}\NormalTok{),}\AttributeTok{end =} \FunctionTok{c}\NormalTok{(}\DecValTok{2024}\NormalTok{), }\AttributeTok{frequency =} \DecValTok{1}\NormalTok{)}
\NormalTok{tempdata}\SpecialCharTok{$}\NormalTok{SON }\OtherTok{\textless{}{-}} \FunctionTok{ts}\NormalTok{(tempdata}\SpecialCharTok{$}\NormalTok{SON, }
  \AttributeTok{start =} \FunctionTok{c}\NormalTok{(}\DecValTok{1880}\NormalTok{),}\AttributeTok{end =} \FunctionTok{c}\NormalTok{(}\DecValTok{2024}\NormalTok{), }\AttributeTok{frequency =} \DecValTok{1}\NormalTok{)}
\NormalTok{tempdata}\SpecialCharTok{$}\NormalTok{J.D }\OtherTok{\textless{}{-}} \FunctionTok{ts}\NormalTok{(tempdata}\SpecialCharTok{$}\NormalTok{J.D, }
  \AttributeTok{start =} \FunctionTok{c}\NormalTok{(}\DecValTok{1880}\NormalTok{),}\AttributeTok{end =} \FunctionTok{c}\NormalTok{(}\DecValTok{2024}\NormalTok{), }\AttributeTok{frequency =} \DecValTok{1}\NormalTok{)}
\end{Highlighting}
\end{Shaded}

Note that we placed each of these quarterly series in the relevant
middle month. You could do the same for the remaining series, but we
will only use the series above in this R walk-through.

We can now use these variables to draw line charts using the
\texttt{plot} function. As an example, we will draw a line chart using
data for January (\texttt{tempdata\$Jan}) for the years 1880--2016. The
\texttt{title} option on the next line adds a chart title, and the
\texttt{abline} option draws a horizontal line according to our
specifications. Make sure to amend the code below so that your chart
title corresponds to the latest year in your dataset (our example uses
2016).

\begin{Shaded}
\begin{Highlighting}[]
\FunctionTok{plot}\NormalTok{(tempdata}\SpecialCharTok{$}\NormalTok{Jan, }\AttributeTok{type =} \StringTok{"l"}\NormalTok{, }\AttributeTok{col =} \StringTok{"blue"}\NormalTok{, }\AttributeTok{lwd =} \DecValTok{2}\NormalTok{,}
     \AttributeTok{ylab =} \StringTok{"Annual temperature anomalies"}\NormalTok{, }\AttributeTok{xlab =} \StringTok{"Year"}\NormalTok{,}
     \AttributeTok{main =} \StringTok{"Average temperature anomaly in January in the northern hemisphere (1880{-}2024)"}\NormalTok{) }\CommentTok{\# Agrega el título aquí}

\CommentTok{\# Agrega la línea horizontal}
\FunctionTok{abline}\NormalTok{(}\AttributeTok{h =} \DecValTok{0}\NormalTok{, }\AttributeTok{col =} \StringTok{"darkorange2"}\NormalTok{, }\AttributeTok{lwd =} \DecValTok{2}\NormalTok{)}

\CommentTok{\# Agrega el texto explicativo}
\FunctionTok{text}\NormalTok{(}\DecValTok{2000}\NormalTok{, }\SpecialCharTok{{-}}\FloatTok{0.1}\NormalTok{, }\StringTok{"1880{-}2024 average"}\NormalTok{)}
\end{Highlighting}
\end{Shaded}

\includegraphics{doing-economics-markdown-working-in-r-project-est--1-_files/figure-latex/unnamed-chunk-6-1.pdf}

Try different values for \texttt{type} and \texttt{col} in the
\texttt{plot} function to figure out what these options do (some online
research could help). \texttt{xlab} and \texttt{ylab} define the
respective axis titles.

It is important to remember that all axis and chart titles should be
enclosed in quotation marks (\texttt{""}), as well as any words that are
not options (for example, colour names or filenames).

{[}End of walk-through{]}

\subsubsection{R walk-through 1.3 Producing a line chart for the annual
temperature
anomalies}\label{r-walk-through-1.3-producing-a-line-chart-for-the-annual-temperature-anomalies}

This is where the power of programming languages becomes evident: to
produce the same line chart for a different variable, we simply take the
code used in R walk-through 1.2 and replace the variable name
\texttt{Jan} with the name for the annual variable (\texttt{J.D}).
Again, make sure to amend the code so that your chart title corresponds
to the latest year in your data (our example uses 2016).

\begin{Shaded}
\begin{Highlighting}[]
\CommentTok{\# Set line width and colour}
\FunctionTok{plot}\NormalTok{(tempdata}\SpecialCharTok{$}\NormalTok{J.D, }\AttributeTok{type =} \StringTok{"l"}\NormalTok{, }\AttributeTok{col =} \StringTok{"blue"}\NormalTok{, }\AttributeTok{lwd =} \DecValTok{2}\NormalTok{,}
  \AttributeTok{ylab =} \StringTok{"Annual temperature anomalies"}\NormalTok{, }\AttributeTok{xlab =} \StringTok{"Year"}\NormalTok{)}

\CommentTok{\# \textbackslash{}n creates a line break}
\FunctionTok{title}\NormalTok{(}\StringTok{"Average annual temperature anomaly }\SpecialCharTok{\textbackslash{}n}\StringTok{ in the northern hemisphere (1880{-}2016)"}\NormalTok{)}

\CommentTok{\# Add a horizontal line (at y = 0)}
\FunctionTok{abline}\NormalTok{(}\AttributeTok{h =} \DecValTok{0}\NormalTok{, }\AttributeTok{col =} \StringTok{"darkorange2"}\NormalTok{, }\AttributeTok{lwd =} \DecValTok{2}\NormalTok{)}

\CommentTok{\# Add a label to the horizontal line}
\FunctionTok{text}\NormalTok{(}\DecValTok{2000}\NormalTok{, }\SpecialCharTok{{-}}\FloatTok{0.1}\NormalTok{, }\StringTok{"1951{-}1980 average"}\NormalTok{)}
\end{Highlighting}
\end{Shaded}

\includegraphics{doing-economics-markdown-working-in-r-project-est--1-_files/figure-latex/unnamed-chunk-7-1.pdf}

{[}End of walk-through{]}

\subsection{Part 1.2 Variation in temperature over
time}\label{part-1.2-variation-in-temperature-over-time}

\subsubsection{R walk-through 1.4 Creating frequency tables and
histograms}\label{r-walk-through-1.4-creating-frequency-tables-and-histograms}

Since we will be looking at data from different subperiods (year
intervals) separately, we will create a categorical variable (a variable
that has two or more categories) that indicates the subperiod for each
observation (row). In R this type of variable is called a `factor
variable'. When we create a factor variable, we need to define the
categories that this variable can take.

\begin{Shaded}
\begin{Highlighting}[]
\NormalTok{tempdata}\SpecialCharTok{$}\NormalTok{Period }\OtherTok{\textless{}{-}} 
  \FunctionTok{factor}\NormalTok{(}\ConstantTok{NA}\NormalTok{, }\AttributeTok{levels =} 
    \FunctionTok{c}\NormalTok{(}\StringTok{"1921{-}1950"}\NormalTok{, }\StringTok{"1951{-}1980"}\NormalTok{, }\StringTok{"1981{-}2010"}\NormalTok{), }
    \AttributeTok{ordered =} \ConstantTok{TRUE}\NormalTok{)}
\end{Highlighting}
\end{Shaded}

We created a new variable called \texttt{Period} and defined the
possible categories (which R refers to as `levels'). Since we will not
be using data for some years (before 1921 and after 2010), we want
\texttt{Period} to take the value `NA' (not available) for these
observations (rows), and the appropriate category for all the other
observations (between 1921--2010). One way to do this is by defining
\texttt{Period} as `NA' for all observations, then change the values of
\texttt{Period} for the observations in 1921--2010.

\begin{Shaded}
\begin{Highlighting}[]
\NormalTok{tempdata}\SpecialCharTok{$}\NormalTok{Period[(tempdata}\SpecialCharTok{$}\NormalTok{Year }\SpecialCharTok{\textgreater{}} \DecValTok{1920}\NormalTok{) }\SpecialCharTok{\&}
\NormalTok{  (tempdata}\SpecialCharTok{$}\NormalTok{Year }\SpecialCharTok{\textless{}} \DecValTok{1951}\NormalTok{)] }\OtherTok{\textless{}{-}} \StringTok{"1921{-}1950"}
\NormalTok{tempdata}\SpecialCharTok{$}\NormalTok{Period[(tempdata}\SpecialCharTok{$}\NormalTok{Year }\SpecialCharTok{\textgreater{}} \DecValTok{1950}\NormalTok{) }\SpecialCharTok{\&}
\NormalTok{  (tempdata}\SpecialCharTok{$}\NormalTok{Year }\SpecialCharTok{\textless{}} \DecValTok{1981}\NormalTok{)] }\OtherTok{\textless{}{-}} \StringTok{"1951{-}1980"}
\NormalTok{tempdata}\SpecialCharTok{$}\NormalTok{Period[(tempdata}\SpecialCharTok{$}\NormalTok{Year }\SpecialCharTok{\textgreater{}} \DecValTok{1980}\NormalTok{) }\SpecialCharTok{\&}
\NormalTok{  (tempdata}\SpecialCharTok{$}\NormalTok{Year }\SpecialCharTok{\textless{}} \DecValTok{2011}\NormalTok{)] }\OtherTok{\textless{}{-}} \StringTok{"1981{-}2010"}
\end{Highlighting}
\end{Shaded}

We need to use all monthly anomalies from June, July, and August, but
they are currently in three separate columns. We will use the \texttt{c}
(combine) function to create one new variable (called
\texttt{temp\_summer}) that contains all these values.

\begin{Shaded}
\begin{Highlighting}[]
\CommentTok{\# Combine the temperature data for June, July, and August}
\NormalTok{temp\_summer }\OtherTok{\textless{}{-}} \FunctionTok{c}\NormalTok{(tempdata}\SpecialCharTok{$}\NormalTok{Jun, tempdata}\SpecialCharTok{$}\NormalTok{Jul, tempdata}\SpecialCharTok{$}\NormalTok{Aug)}
\end{Highlighting}
\end{Shaded}

Now we have one long variable (\texttt{temp\_summer}), with the monthly
temperature anomalies for the three months (from 1880 to the latest
year) attached to each other. But remember that we want to make separate
calculations for each category in \texttt{Period} (1921--1950,
1951--1980, 1981--2010). To make a variable showing the categories for
the \texttt{temp\_summer} variable, we use the \texttt{c} function
again.

\begin{Shaded}
\begin{Highlighting}[]
\NormalTok{temp\_summer }\OtherTok{\textless{}{-}} \FunctionTok{unlist}\NormalTok{(tempdata[,}\DecValTok{7}\SpecialCharTok{:}\DecValTok{9}\NormalTok{],}\AttributeTok{use.names =}\ConstantTok{FALSE}\NormalTok{)}
\end{Highlighting}
\end{Shaded}

\begin{Shaded}
\begin{Highlighting}[]
\CommentTok{\# Mirror the Period information for temp\_sum}
\NormalTok{temp\_Period }\OtherTok{\textless{}{-}} 
\FunctionTok{c}\NormalTok{(tempdata}\SpecialCharTok{$}\NormalTok{Period, tempdata}\SpecialCharTok{$}\NormalTok{Period, tempdata}\SpecialCharTok{$}\NormalTok{Period)}

\CommentTok{\# Repopulate the factor information }
\end{Highlighting}
\end{Shaded}

After using the \texttt{c} function, we had to use the \texttt{factor}
function again to tell R that our new variable \texttt{temp\_Period} is
a factor variable.

We have now created the variables needed to make frequency tables and
histograms (\texttt{temp\_summer} and \texttt{temp\_Period}). To obtain
the frequency table for 1951--1980, we use the \texttt{hist} function on
the monthly temperature anomalies from the period `1951--1980':
\texttt{temp\_summer{[}(temp\_Period\ ==\ "1951-1980"){]}}. The option
\texttt{plot\ =\ FALSE} tells R not to make a plot of this information.
(See what happens if you set it to \texttt{TRUE}.)

\begin{Shaded}
\begin{Highlighting}[]
\FunctionTok{hist}\NormalTok{(temp\_summer[(temp\_Period }\SpecialCharTok{==} \StringTok{"1951{-}1980"}\NormalTok{)])}
\end{Highlighting}
\end{Shaded}

\includegraphics{doing-economics-markdown-working-in-r-project-est--1-_files/figure-latex/unnamed-chunk-13-1.pdf}

From the output you can see that we can get the temperature ranges (the
values in \texttt{\$breaks} correspond to Column 1 of Figure 1.5) and
the frequencies (\texttt{\$counts}), which is all we need to create a
frequency table. However, in our case the frequency table is merely a
temporary input required to produce a histogram.

We can make the three histograms we need all at once, using the
\texttt{histogram} function from the \texttt{mosaic} package.

The function below includes multiple commands:

\begin{itemize}
\tightlist
\item
  \texttt{\textbar{}\ temp\_Period} splits the data according to its
  category, given by \texttt{temp\_Period}.
\item
  \texttt{type\ =\ "count"} indicates that we want to display the counts
  (frequencies) in each category.
\item
  \texttt{breaks\ =\ seq(-0.5,\ 1.3,\ 0.1)} gives a sequence of numbers
  −0.5, −0.4,~\ldots, 1.3, which are boundaries for the categories.
\item
  \texttt{main\ =\ "Histogram\ of\ temperature\ anomalies"} gives Figure
  1.6 its title.
\end{itemize}

\begin{Shaded}
\begin{Highlighting}[]
\CommentTok{\# Load the library we use for the following command.}
\FunctionTok{library}\NormalTok{(mosaic)}
\end{Highlighting}
\end{Shaded}

\begin{verbatim}
## Registered S3 method overwritten by 'mosaic':
##   method                           from   
##   fortify.SpatialPolygonsDataFrame ggplot2
\end{verbatim}

\begin{verbatim}
## 
## The 'mosaic' package masks several functions from core packages in order to add 
## additional features.  The original behavior of these functions should not be affected by this.
\end{verbatim}

\begin{verbatim}
## 
## Adjuntando el paquete: 'mosaic'
\end{verbatim}

\begin{verbatim}
## The following objects are masked from 'package:dplyr':
## 
##     count, do, tally
\end{verbatim}

\begin{verbatim}
## The following object is masked from 'package:Matrix':
## 
##     mean
\end{verbatim}

\begin{verbatim}
## The following object is masked from 'package:ggplot2':
## 
##     stat
\end{verbatim}

\begin{verbatim}
## The following objects are masked from 'package:stats':
## 
##     binom.test, cor, cor.test, cov, fivenum, IQR, median, prop.test,
##     quantile, sd, t.test, var
\end{verbatim}

\begin{verbatim}
## The following objects are masked from 'package:base':
## 
##     max, mean, min, prod, range, sample, sum
\end{verbatim}

\begin{Shaded}
\begin{Highlighting}[]
\FunctionTok{histogram}\NormalTok{(}\SpecialCharTok{\textasciitilde{}}\NormalTok{ temp\_summer }\SpecialCharTok{|}\NormalTok{ temp\_Period, }\AttributeTok{type =} \StringTok{"count"}\NormalTok{, }
  \AttributeTok{breaks =} \FunctionTok{seq}\NormalTok{(}\SpecialCharTok{{-}}\FloatTok{0.5}\NormalTok{, }\FloatTok{1.3}\NormalTok{, }\FloatTok{0.10}\NormalTok{), }
  \AttributeTok{main =} \StringTok{"Histogram of Temperature anomalies"}\NormalTok{, }
  \AttributeTok{xlab =} \StringTok{"Summer temperature distribution"}\NormalTok{)}
\end{Highlighting}
\end{Shaded}

\includegraphics{doing-economics-markdown-working-in-r-project-est--1-_files/figure-latex/unnamed-chunk-14-1.pdf}

\subsubsection{\texorpdfstring{R walk-through 1.5 Using the
\texttt{quantile}
function}{R walk-through 1.5 Using the quantile function}}\label{r-walk-through-1.5-using-the-quantile-function}

First, we need to create a variable that contains all monthly anomalies
in the years 1951--1980. Then, we use R's \texttt{quantile} function to
find the required percentiles (0.3 and 0.7 refer to the 3rd and 7th
deciles, respectively).

\emph{Note}: You may get slightly different values to those shown here
if you are using the latest data.

\begin{Shaded}
\begin{Highlighting}[]
\CommentTok{\# Select years 1951 to 1980}
\NormalTok{temp\_all\_months }\OtherTok{\textless{}{-}} \FunctionTok{subset}\NormalTok{(tempdata, }
\NormalTok{  (Year }\SpecialCharTok{\textgreater{}=} \DecValTok{1951} \SpecialCharTok{\&}\NormalTok{ Year }\SpecialCharTok{\textless{}=} \DecValTok{1980}\NormalTok{))}
                   
\CommentTok{\# Columns 2 to 13 contain months Jan to Dec.}
\NormalTok{temp\_51to80 }\OtherTok{\textless{}{-}} \FunctionTok{unlist}\NormalTok{(temp\_all\_months[, }\DecValTok{2}\SpecialCharTok{:}\DecValTok{13}\NormalTok{])}
      
\CommentTok{\# c(0.3, 0.7) indicates the chosen percentiles.}
\NormalTok{perc }\OtherTok{\textless{}{-}} \FunctionTok{quantile}\NormalTok{(temp\_51to80, }\FunctionTok{c}\NormalTok{(}\FloatTok{0.3}\NormalTok{, }\FloatTok{0.7}\NormalTok{))   }

\CommentTok{\# The cold threshold}
\NormalTok{p30 }\OtherTok{\textless{}{-}}\NormalTok{ perc[}\DecValTok{1}\NormalTok{]}
\NormalTok{p30}
\end{Highlighting}
\end{Shaded}

\begin{verbatim}
##  30% 
## -0.1
\end{verbatim}

\begin{Shaded}
\begin{Highlighting}[]
\CommentTok{\# The hot threshold}
\NormalTok{p70 }\OtherTok{\textless{}{-}}\NormalTok{ perc[}\DecValTok{2}\NormalTok{]}
\NormalTok{p70}
\end{Highlighting}
\end{Shaded}

\begin{verbatim}
## 70% 
## 0.1
\end{verbatim}

\emph{Explica el resultado}

{[}End of walk-through{]}

\subsubsection{\texorpdfstring{R walk-through 1.6 Using the
\texttt{mean}
function}{R walk-through 1.6 Using the mean function}}\label{r-walk-through-1.6-using-the-mean-function}

\emph{Note}: You may get slightly different values to those shown here
if you are using the latest data.

We repeat the steps used in R walk-through 1.5, now looking at monthly
anomalies in the years 1981--2010. We can simply change the year values
in the code from R walk-through 1.5.

\begin{Shaded}
\begin{Highlighting}[]
\CommentTok{\# Select years 1951 to 1980}
\NormalTok{temp\_all\_months }\OtherTok{\textless{}{-}} \FunctionTok{subset}\NormalTok{(tempdata, }
\NormalTok{  (Year }\SpecialCharTok{\textgreater{}=} \DecValTok{1981} \SpecialCharTok{\&}\NormalTok{ Year }\SpecialCharTok{\textless{}=} \DecValTok{2010}\NormalTok{))}
                   
\CommentTok{\# Columns 2 to 13 contain months Jan to Dec.}
\NormalTok{temp\_81to10 }\OtherTok{\textless{}{-}} \FunctionTok{unlist}\NormalTok{(temp\_all\_months[, }\DecValTok{2}\SpecialCharTok{:}\DecValTok{13}\NormalTok{])}
\end{Highlighting}
\end{Shaded}

Now that we have all the monthly data for 1981--2010, we want to count
the proportion of observations that are smaller than --0.1. This is
easily achieved with the following lines of code:

\begin{Shaded}
\begin{Highlighting}[]
\FunctionTok{paste}\NormalTok{(}\StringTok{"Proportion smaller than p30"}\NormalTok{)}
\end{Highlighting}
\end{Shaded}

\begin{verbatim}
## [1] "Proportion smaller than p30"
\end{verbatim}

\begin{Shaded}
\begin{Highlighting}[]
\NormalTok{temp }\OtherTok{\textless{}{-}}\NormalTok{ temp\_81to10 }\SpecialCharTok{\textless{}}\NormalTok{ p30}
\FunctionTok{mean}\NormalTok{(temp)}
\end{Highlighting}
\end{Shaded}

\begin{verbatim}
## [1] 0.01944444
\end{verbatim}

\emph{Explica el resultado}

Let's check whether we get a similar result for the number of
observations that are larger than 0.11.

\begin{Shaded}
\begin{Highlighting}[]
\FunctionTok{paste}\NormalTok{(}\StringTok{"Proportion larger than p70"}\NormalTok{)}
\end{Highlighting}
\end{Shaded}

\begin{verbatim}
## [1] "Proportion larger than p70"
\end{verbatim}

\begin{Shaded}
\begin{Highlighting}[]
\FunctionTok{mean}\NormalTok{(temp\_81to10 }\SpecialCharTok{\textgreater{}}\NormalTok{ p70)}
\end{Highlighting}
\end{Shaded}

\begin{verbatim}
## [1] 0.8472222
\end{verbatim}

{[}End of walk-through{]}

\subsubsection{R walk-through 1.7 Calculating and understanding mean and
variance}\label{r-walk-through-1.7-calculating-and-understanding-mean-and-variance}

Calculate mean and variance. One option is to use the \texttt{mosaic}
package

\begin{Shaded}
\begin{Highlighting}[]
\FunctionTok{paste}\NormalTok{(}\StringTok{"Mean of DJF temperature anomalies across periods"}\NormalTok{)}
\end{Highlighting}
\end{Shaded}

\begin{verbatim}
## [1] "Mean of DJF temperature anomalies across periods"
\end{verbatim}

\begin{Shaded}
\begin{Highlighting}[]
\FunctionTok{mean}\NormalTok{(}\SpecialCharTok{\textasciitilde{}}\NormalTok{DJF}\SpecialCharTok{|}\NormalTok{Period,}\AttributeTok{data =}\NormalTok{ tempdata)}
\end{Highlighting}
\end{Shaded}

\begin{verbatim}
##    1921-1950    1951-1980    1981-2010 
## -0.030333333 -0.002666667  0.523333333
\end{verbatim}

\begin{Shaded}
\begin{Highlighting}[]
\FunctionTok{paste}\NormalTok{(}\StringTok{"Variance of DJF anomalies across periods"}\NormalTok{)}
\end{Highlighting}
\end{Shaded}

\begin{verbatim}
## [1] "Variance of DJF anomalies across periods"
\end{verbatim}

\begin{Shaded}
\begin{Highlighting}[]
\FunctionTok{var}\NormalTok{(}\SpecialCharTok{\textasciitilde{}}\NormalTok{DJF}\SpecialCharTok{|}\NormalTok{Period,}\AttributeTok{data =}\NormalTok{ tempdata)}
\end{Highlighting}
\end{Shaded}

\begin{verbatim}
##  1921-1950  1951-1980  1981-2010 
## 0.05672057 0.05038575 0.07871264
\end{verbatim}

Using the data in tempdata (\texttt{data\ =\ tempdata}), we calculated
the mean (\texttt{mean}) and variance (\texttt{var}) of variable
\texttt{\textasciitilde{}DJF} separately for (\texttt{\textbar{}}) each
value of \texttt{Period}. The \texttt{mosaic} package allows us to
calculate the means/variances for each period all at once. If
\texttt{mosaic} is not loaded, you will get the error message:
\texttt{Error\ in\ mean(\textasciitilde{}DJF\ \textbackslash{}\textbar{}\ Period,\ data\ =\ tempdata)\ :\ unused\ argument\ (data\ =\ tempdata)}.

\emph{Interpreta los resultados}

Let's calculate the variances through the periods for the other seasons.

\begin{Shaded}
\begin{Highlighting}[]
\FunctionTok{paste}\NormalTok{(}\StringTok{"Variance of MAM anomalies across periods"}\NormalTok{)}
\end{Highlighting}
\end{Shaded}

\begin{verbatim}
## [1] "Variance of MAM anomalies across periods"
\end{verbatim}

\begin{Shaded}
\begin{Highlighting}[]
\FunctionTok{var}\NormalTok{(}\SpecialCharTok{\textasciitilde{}}\NormalTok{MAM}\SpecialCharTok{|}\NormalTok{Period,}\AttributeTok{data =}\NormalTok{ tempdata)}
\end{Highlighting}
\end{Shaded}

\begin{verbatim}
##  1921-1950  1951-1980  1981-2010 
## 0.03099782 0.02540000 0.07573345
\end{verbatim}

\begin{Shaded}
\begin{Highlighting}[]
\FunctionTok{paste}\NormalTok{(}\StringTok{"Variance of JJA anomalies across periods"}\NormalTok{)}
\end{Highlighting}
\end{Shaded}

\begin{verbatim}
## [1] "Variance of JJA anomalies across periods"
\end{verbatim}

\begin{Shaded}
\begin{Highlighting}[]
\FunctionTok{var}\NormalTok{(}\SpecialCharTok{\textasciitilde{}}\NormalTok{JJA}\SpecialCharTok{|}\NormalTok{Period,}\AttributeTok{data =}\NormalTok{ tempdata)}
\end{Highlighting}
\end{Shaded}

\begin{verbatim}
##  1921-1950  1951-1980  1981-2010 
## 0.02128920 0.01460644 0.06749609
\end{verbatim}

\begin{Shaded}
\begin{Highlighting}[]
\FunctionTok{paste}\NormalTok{(}\StringTok{"Variance of SON anomalies across periods"}\NormalTok{)}
\end{Highlighting}
\end{Shaded}

\begin{verbatim}
## [1] "Variance of SON anomalies across periods"
\end{verbatim}

\begin{Shaded}
\begin{Highlighting}[]
\FunctionTok{var}\NormalTok{(}\SpecialCharTok{\textasciitilde{}}\NormalTok{SON}\SpecialCharTok{|}\NormalTok{Period,}\AttributeTok{data =}\NormalTok{ tempdata)}
\end{Highlighting}
\end{Shaded}

\begin{verbatim}
##  1921-1950  1951-1980  1981-2010 
## 0.02819264 0.02635126 0.11104644
\end{verbatim}

\emph{Interpreta lso resutlados}

We can plot a line chart to see these changes graphically. (This type of
chart is formally known as a `time-series plot'). Make sure to change
the chart title according to the latest year in your data (here we used
2016).

\begin{Shaded}
\begin{Highlighting}[]
\FunctionTok{plot}\NormalTok{(tempdata}\SpecialCharTok{$}\NormalTok{DJF, }\AttributeTok{type =} \StringTok{"l"}\NormalTok{, }\AttributeTok{col =} \StringTok{"blue"}\NormalTok{, }\AttributeTok{lwd =} \DecValTok{2}\NormalTok{,}
  \AttributeTok{ylab =} \StringTok{"Annual temperature anomalies"}\NormalTok{, }\AttributeTok{xlab =} \StringTok{"Year"}\NormalTok{)}

\CommentTok{\# \textbackslash{}n creates a line break}
\FunctionTok{title}\NormalTok{(}\StringTok{"Average temperature anomaly in DJF and JJA }\SpecialCharTok{\textbackslash{}n}\StringTok{ in the northern hemisphere (1880{-}2016)"}\NormalTok{)}

\CommentTok{\# Add a horizontal line (at y = 0)}
\FunctionTok{abline}\NormalTok{(}\AttributeTok{h =} \DecValTok{0}\NormalTok{, }\AttributeTok{col =} \StringTok{"darkorange2"}\NormalTok{, }\AttributeTok{lwd =} \DecValTok{2}\NormalTok{)}
\FunctionTok{lines}\NormalTok{(tempdata}\SpecialCharTok{$}\NormalTok{JJA, }\AttributeTok{col =} \StringTok{"darkgreen"}\NormalTok{, }\AttributeTok{lwd =} \DecValTok{2}\NormalTok{) }

\CommentTok{\# Add a label to the horizontal line}
\FunctionTok{text}\NormalTok{(}\DecValTok{1895}\NormalTok{, }\FloatTok{0.1}\NormalTok{, }\StringTok{"1951{-}1980 average"}\NormalTok{)}
\FunctionTok{legend}\NormalTok{(}\DecValTok{1880}\NormalTok{, }\FloatTok{1.5}\NormalTok{, }\AttributeTok{legend =} \FunctionTok{c}\NormalTok{(}\StringTok{"DJF"}\NormalTok{, }\StringTok{"JJA"}\NormalTok{),}
  \AttributeTok{col =} \FunctionTok{c}\NormalTok{(}\StringTok{"blue"}\NormalTok{, }\StringTok{"darkgreen"}\NormalTok{), }
  \AttributeTok{lty =} \DecValTok{1}\NormalTok{, }\AttributeTok{cex =} \FloatTok{0.8}\NormalTok{, }\AttributeTok{lwd =} \DecValTok{2}\NormalTok{)}
\end{Highlighting}
\end{Shaded}

\includegraphics{doing-economics-markdown-working-in-r-project-est--1-_files/figure-latex/unnamed-chunk-32-1.pdf}

{[}End of walk-through{]}

\subsection{Part 1.3 Carbon emissions and the
environment}\label{part-1.3-carbon-emissions-and-the-environment}

\subsubsection{R walk-through 1.8 Scatterplots and the correlation
coefficient}\label{r-walk-through-1.8-scatterplots-and-the-correlation-coefficient}

First we will use the \texttt{read.csv} function to import the CO2
datafile into R, and call it \texttt{CO2data}.

\begin{Shaded}
\begin{Highlighting}[]
\NormalTok{CO2data }\OtherTok{\textless{}{-}} \FunctionTok{read.csv}\NormalTok{(}\StringTok{"C:/Users/moral/OneDrive/Escritorio/Taller\_3\_Bohada\_Sarmiento\_Lopez\_Arias/RAW/1\_CO2 data.csv"}\NormalTok{)}
\end{Highlighting}
\end{Shaded}

This file has monthly data, but in contrast to the data in
\texttt{tempdata}, the data is all in one column (this is more
conventional than the column per month format). To make this task
easier, we will pick the June data from the CO2 emissions and add them
as an additional variable to the \texttt{tempdata} dataset.

R has a convenient function called \texttt{merge} to do this. First we
create a new dataset that contains only the June emissions data
(`CO2data\_june').

\begin{Shaded}
\begin{Highlighting}[]
\NormalTok{CO2data\_june }\OtherTok{\textless{}{-}}\NormalTok{ CO2data[CO2data}\SpecialCharTok{$}\NormalTok{Month }\SpecialCharTok{==} \DecValTok{6}\NormalTok{,]}
\end{Highlighting}
\end{Shaded}

Then we use this data in the \texttt{merge} function. The \texttt{merge}
function takes the original `tempdata' and the `CO2data' and merges
(combines) them together. As the two dataframes have a common variable,
\texttt{Year}, R automatically matches the data by year.

(\emph{Extension:} Look up \texttt{?merge} or Google `How to use the R
merge function' to figure out what \texttt{all.x} does, and to see other
options that this function allows.)

\begin{Shaded}
\begin{Highlighting}[]
\FunctionTok{names}\NormalTok{(CO2data)[}\DecValTok{1}\NormalTok{] }\OtherTok{\textless{}{-}} \StringTok{"Year"}
\NormalTok{tempCO2data }\OtherTok{\textless{}{-}} \FunctionTok{merge}\NormalTok{(tempdata, CO2data\_june)}
\end{Highlighting}
\end{Shaded}

Let us have a look at the data and check that it was combined correctly:

\begin{Shaded}
\begin{Highlighting}[]
\FunctionTok{head}\NormalTok{(tempCO2data[, }\FunctionTok{c}\NormalTok{(}\StringTok{"Year"}\NormalTok{, }\StringTok{"Jun"}\NormalTok{, }\StringTok{"Trend"}\NormalTok{)])}
\end{Highlighting}
\end{Shaded}

\begin{verbatim}
##   Year   Jun  Trend
## 1 1958  0.05 314.85
## 2 1959  0.14 315.92
## 3 1960  0.18 317.36
## 4 1961  0.18 317.48
## 5 1962 -0.13 318.27
## 6 1963 -0.04 319.16
\end{verbatim}

\begin{verbatim}
##   Year   Jun  Trend
## 1 1958  0.04 314.85
## 2 1959  0.14 315.92
## 3 1960  0.18 317.36
## 4 1961  0.19 317.48
## 5 1962 -0.10 318.27
## 6 1963 -0.02 319.16
\end{verbatim}

To make a scatterplot, we use the \texttt{plot} function. R's default
chart for \texttt{plot} is a scatterplot, so we do not need to specify
the chart type. One new option that applies to scatterplots is
\texttt{pch\ =}, which determines the appearance of the data points. The
number 16 corresponds to filled-in circles, but you can experiment with
other numbers (from 0 to 25) to see what the data points look like.

\begin{Shaded}
\begin{Highlighting}[]
\FunctionTok{plot}\NormalTok{(tempCO2data}\SpecialCharTok{$}\NormalTok{Jun, tempCO2data}\SpecialCharTok{$}\NormalTok{Trend, }
  \AttributeTok{xlab =} \StringTok{"Temperature anomaly (degrees Celsius)"}\NormalTok{, }
  \AttributeTok{ylab =} \StringTok{"CO2 levels (trend, mole fraction)"}\NormalTok{, }
  \AttributeTok{pch =} \DecValTok{16}\NormalTok{, }\AttributeTok{col =} \StringTok{"blue"}\NormalTok{)}

\FunctionTok{title}\NormalTok{(}\StringTok{"Scatterplot for CO2 emissions and temperature anomalies"}\NormalTok{)}
\end{Highlighting}
\end{Shaded}

\includegraphics{doing-economics-markdown-working-in-r-project-est--1-_files/figure-latex/unnamed-chunk-37-1.pdf}

The \texttt{cor} function calculates the correlation coefficient.
\emph{Note}: You may get slightly different results if you are using the
latest data.

\begin{Shaded}
\begin{Highlighting}[]
\FunctionTok{cor}\NormalTok{(tempCO2data}\SpecialCharTok{$}\NormalTok{Jun, tempCO2data}\SpecialCharTok{$}\NormalTok{Trend)}
\end{Highlighting}
\end{Shaded}

\begin{verbatim}
## [1] 0.9149093
\end{verbatim}

\begin{verbatim}
## [1] 0.9157744
\end{verbatim}

\emph{Interpretar resultados}

One limitation of this correlation measure is that it only tells us
about the strength of the upward- or downward-sloping linear
relationship between two variables, in other words how closely the
scatterplot aligns along an upward- or downward-sloping straight line.
The correlation coefficient cannot tell us if the two variables have a
different kind of relationship (such as that represented by a wavy
line).

\emph{Note:} The word `strong' is used for coefficients that are close
to 1 or −1, and `weak' is used for coefficients that are close to 0,
though there is no precise range of values that are considered `strong'
or `weak'.

If you need more insight into correlation coefficients, you may find it
helpful to watch online tutorials such as `Correlation coefficient
intuition' (\url{https://tinyco.re/4363520}) from the Khan Academy.

As we are dealing with time-series data, it is often more instructive to
look at a line plot, as a scatterplot cannot convey how the observations
relate to each other in the time dimension. If you were to check the
variable types (using \texttt{str(tempCO2data)}), you would see that the
data is not yet in time-series format. We could continue with the format
as it is, but for plotting purposes it is useful to let R know that we
are dealing with time-series data. We therefore apply the \texttt{ts}
function as we did in Part 1.1.

\begin{Shaded}
\begin{Highlighting}[]
\NormalTok{tempCO2data}\SpecialCharTok{$}\NormalTok{Jun }\OtherTok{\textless{}{-}} \FunctionTok{ts}\NormalTok{(tempCO2data}\SpecialCharTok{$}\NormalTok{Jun, }
  \AttributeTok{start =} \FunctionTok{c}\NormalTok{(}\DecValTok{1958}\NormalTok{), }\AttributeTok{end =} \FunctionTok{c}\NormalTok{(}\DecValTok{2017}\NormalTok{), }\AttributeTok{frequency =} \DecValTok{1}\NormalTok{) }
\NormalTok{tempCO2data}\SpecialCharTok{$}\NormalTok{Trend }\OtherTok{\textless{}{-}} \FunctionTok{ts}\NormalTok{(tempCO2data}\SpecialCharTok{$}\NormalTok{Trend, }
  \AttributeTok{start =} \FunctionTok{c}\NormalTok{(}\DecValTok{1958}\NormalTok{), }\AttributeTok{end =} \FunctionTok{c}\NormalTok{(}\DecValTok{2017}\NormalTok{), }\AttributeTok{frequency =} \DecValTok{1}\NormalTok{) }
\end{Highlighting}
\end{Shaded}

Let's start by plotting the June temperature anomalies.

\begin{Shaded}
\begin{Highlighting}[]
\FunctionTok{plot}\NormalTok{(tempCO2data}\SpecialCharTok{$}\NormalTok{Jun, }\AttributeTok{type =} \StringTok{"l"}\NormalTok{, }\AttributeTok{col =} \StringTok{"blue"}\NormalTok{, }\AttributeTok{lwd =} \DecValTok{2}\NormalTok{,}
  \AttributeTok{ylab =} \StringTok{"June temperature anomalies"}\NormalTok{, }\AttributeTok{xlab =} \StringTok{"Year"}\NormalTok{)}

\FunctionTok{title}\NormalTok{(}\StringTok{"June temperature anomalies and CO2 emissions"}\NormalTok{)   }
\end{Highlighting}
\end{Shaded}

\includegraphics{doing-economics-markdown-working-in-r-project-est--1-_files/figure-latex/unnamed-chunk-40-1.pdf}

Typically, when using the \texttt{plot} function we would now only need
to add the line for the second variable using the \texttt{lines}
command. The issue, however, is that the CO2 emissions variable
(\texttt{Trend}) is on a different scale, and the automatic vertical
axis scale (from --0.2 to about 1.2) would not allow for the display of
\texttt{Trend}. To resolve this issue you can introduce a second
vertical axis using the commands below. (\emph{Tip:} You are unlikely to
remember the exact commands required, however you can Google `R plot 2
vertical axes' or a similar search term, and then adjust the code you
find so it will work on your dataset.)

\begin{Shaded}
\begin{Highlighting}[]
\FunctionTok{par}\NormalTok{(}\AttributeTok{mar =} \FunctionTok{c}\NormalTok{(}\DecValTok{5}\NormalTok{, }\DecValTok{5}\NormalTok{, }\DecValTok{2}\NormalTok{, }\DecValTok{5}\NormalTok{))}

\FunctionTok{plot}\NormalTok{(tempCO2data}\SpecialCharTok{$}\NormalTok{Jun, }\AttributeTok{type =} \StringTok{"l"}\NormalTok{, }\AttributeTok{col =} \StringTok{"blue"}\NormalTok{, }\AttributeTok{lwd =} \DecValTok{2}\NormalTok{,}
  \AttributeTok{ylab =} \StringTok{"June temperature anomalies"}\NormalTok{, }\AttributeTok{xlab =} \StringTok{"Year"}\NormalTok{)}

\FunctionTok{title}\NormalTok{(}\StringTok{"June temperature anomalies and CO2 emissions"}\NormalTok{)  }

\CommentTok{\# This puts the next plot into the same picture.}
\FunctionTok{par}\NormalTok{(}\AttributeTok{new =}\NormalTok{ T)}

\CommentTok{\# No axis, no labels}
\FunctionTok{plot}\NormalTok{(tempCO2data}\SpecialCharTok{$}\NormalTok{Trend, }\AttributeTok{pch =} \DecValTok{16}\NormalTok{, }\AttributeTok{lwd =} \DecValTok{2}\NormalTok{, }
  \AttributeTok{axes =} \ConstantTok{FALSE}\NormalTok{, }\AttributeTok{xlab =} \ConstantTok{NA}\NormalTok{, }\AttributeTok{ylab =} \ConstantTok{NA}\NormalTok{, }\AttributeTok{cex =} \FloatTok{1.2}\NormalTok{) }
\FunctionTok{axis}\NormalTok{(}\AttributeTok{side =} \DecValTok{4}\NormalTok{)}
\FunctionTok{mtext}\NormalTok{(}\AttributeTok{side =} \DecValTok{4}\NormalTok{, }\AttributeTok{line =} \DecValTok{3}\NormalTok{, }\StringTok{\textquotesingle{}CO2 emissions\textquotesingle{}}\NormalTok{)}

\FunctionTok{legend}\NormalTok{(}\StringTok{"topleft"}\NormalTok{, }\AttributeTok{legend =} \FunctionTok{c}\NormalTok{(}\StringTok{"June temp anom"}\NormalTok{, }\StringTok{"CO2 emis"}\NormalTok{),}
  \AttributeTok{lty =} \FunctionTok{c}\NormalTok{(}\DecValTok{1}\NormalTok{, }\DecValTok{1}\NormalTok{), }\AttributeTok{col =} \FunctionTok{c}\NormalTok{(}\StringTok{"blue"}\NormalTok{, }\StringTok{"black"}\NormalTok{), }\AttributeTok{lwd =} \DecValTok{2}\NormalTok{)}
\end{Highlighting}
\end{Shaded}

\includegraphics{doing-economics-markdown-working-in-r-project-est--1-_files/figure-latex/unnamed-chunk-41-1.pdf}

\emph{Interpreta los resultados}

{[}End of walk-through{]}

\end{document}
